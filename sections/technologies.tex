\section{Mqtt}
\setauthor{Stefan Schwammal}

\section{Keykloak}
\setauthor{Elias Mahr}
Keycloak ist ein Open Source System für Identität und Zugriff, das die sichere Anmeldung und die Rechteverwaltung für unsere Anwendungen übernimmt. Die Anwendungen schicken die Anmeldung an Keycloak und müssen das Speichern und das Prüfen von Benutzerkonten nicht mehr selbst erledigen. So kann die Anwendung sich auf das Wesentliche konzentrieren, während Keycloak die Anmeldung und die Rechte im Hintergrund mit Sicherheitsstandard für Unternehmen regelt.

\subsection{Grundidee und Zweck}
Die Firma Herbsthofer legt sehr großen Wert auf Sicherheit. Die externe Bedienung von Heizung, Kühlung und Beleuchtung kann, wenn es ausgenutzt wird zu erheblichen Schäden führen.

\begin{itemize}
\item Mitarbeiterschutz:
Bei voll eingeschalteter Kühlung an kalten Wintertagen kann es zu Gesundheitlichen Schäden in Form von verkühlungen bei Arbeitern/Arbeiterinnen führen, die auch mit einem Ausfall für ein paar Tage enden können und so der Firma Geld kosten.
\item Brandgefahr:
Nicht zu vergessen ist auch die erhöhte Brandgefahr. Unkontrollierte Heizbetriebe auf einer zu hohen Temperaturstufe kann im Schlimmsten Fall zu Bränden führen und so erheblichen Schaden anrichten und Personenschaden mit sich führen.
\item Kosten: 
Strom wird immer Teurer. Bei durchgehender Nutzung auch an Tagen an denen niemand im Container ist, ist erstens der Schaden an der Umwelt durch extreme Ressourcen Verschwändung zu Beachten. Jedoch auch die immensen, unnötigen Kosten die, die Firma bei Ihrer Stromrechnung erwarten.
\end{itemize}
