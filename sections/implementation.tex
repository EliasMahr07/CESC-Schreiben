\section{Systemarchitektur [Elias Mahr]}
\setauthor{Elias Mahr}

\begin{figure}[H]
    \centering
    \includegraphics[scale=0.35]{pics/Systhemarchitektur.png}
    \caption{Systemarchitektur}
    \label{fig:FrontendButton}
\end{figure}

\section{Containerarchitektur [Elias Mahr]}
\setauthor{Elias Mahr}

\begin{figure}[H]
    \centering
    \includegraphics[scale=0.5]{pics/ContainerArchitectur.png}
    \caption{Containerarchitektur}
    \label{fig:FrontendButton}
\end{figure}

\section{Keycloak [Elias Mahr]}
\setauthor{Elias Mahr}

\subsection{Einbindung ins Frontend}

\begin{figure}[H]
    \centering
    \includegraphics[scale=0.5]{pics/FrontendButton.png}
    \caption{Button im Frontend}
    \label{fig:FrontendButton}
\end{figure}

Um die Webapp möglichst unkompliziert und benutzerfreundlich zu halten, ist der Button, wie im Screenshot ersichtlich, rechts oben in der Ecke platziert. Unabhängig vom Scroll-Zustand oder der aktuell ausgewählten Page bleibt er auch immer an dieser Stelle, um ein schnelles, unkompliziertes Anmelden zu gewährleisten.

\begin{figure}[H]
    \centering
    \begin{subfigure}[b]{0.45\textwidth}
        \centering
        \includegraphics[width=0.5\textwidth]{pics/anmelden.png}
        \caption{Anmelde-Button}
        \label{fig:anmeldebutton}
    \end{subfigure}
    \hfill
    \begin{subfigure}[b]{0.45\textwidth}
        \centering
        \includegraphics[width=0.5\textwidth]{pics/abmelden.png}
        \caption{Abmelde-Button}
        \label{fig:abmeldebutton}
    \end{subfigure}
    \caption{Login- und Logout-Buttons}
    \label{fig:loginbuttons}
\end{figure}

Das Icon stammt aus der Font Awesome Bibliothek:
\\
\lstinline[language=HTML]{<i class="fas fa-sign-in-alt"></i>}
\\
Bei Klick darauf wird man weitergeleitet zum Pfad:\\ \texttt{https://192.168.20.202/auth/realms/cesc/protocol/openid-connect\\/auth?client\_id=cesc-frontend\&redirect\_uri=https...} 
Unter dieser Url wird der Standard-Login-Dialog von Keycloak geöffnet. Nach erfolgreicher Anmeldung bekommt der User seinen Token, wird automatisch wieder zur vorher besuchten Seite umgeleitet und es werden alle Funktionen der Webseite freigeschaltet.

\subsection{Der Realm}
Ein Realm in Keycloak ist ein isolierter Sicherheitsbereich. Er umfasst Benutzer, Rollen, Clients und Policies für eine bestimmte Anwendung. Der Realm namens ``cesc'' definiert die gesamte Identifikation und Authorisierung für die Weboberfläche. Zurzeit ist dieser Bereich möglichst klein gehalten, jedoch vollkommen ausreichend. Ein Admin User(herbsthofer) steuert alles, in Zukunft kann man jedoch leicht noch andere Benutzer/Rollen hinzu fügen.
\subsubsection{Realm Einstellungen}

\begin{lstlisting}[
    caption=Keycloak Realm Konfiguration,
    label=lst:realmconfig,
    basicstyle=\ttfamily\small
]
{
  "realm": "cesc",
  "enabled": true,
  "sslRequired": "external",
  "registrationAllowed": false,
  "loginWithEmailAllowed": true,
  "duplicateEmailsAllowed": false,
  "resetPasswordAllowed": true,
  "editUsernameAllowed": false,
  "bruteForceProtected": true
}
\end{lstlisting}
In den wichtigsten Einstellungen wird unter anderem definiert, dass für alle externen Verbindungen SSL vorgeschrieben ist.
Die Zeile \texttt{bruteForceProtected: true} ist eine einfache Absicherung gegen Brute Force Angriffe auf Keycloak. Durch sie wird ein Account nach mehreren fehlgeschlagenen Anmeldungen temporär gesperrt.
Um zukünftigen Admins der Webseite eine leichte und unkomplizierte Anmeldung zu ermöglichen, ist auch das Einloggen mit der E-Mail-Adresse erlaubt.

\subsubsection{Rollen}
\begin{lstlisting}[
    caption=Keycloak Realm Konfiguration,
    label=lst:realmconfig,
    basicstyle=\ttfamily\small
]
{
  "roles": {
    "realm": [
      {
        "name": "admin",
        "description": "Administrator role for CESC application",
        "composite": false,
        "clientRole": false
      }
    ]
  }
}
\end{lstlisting}
Der aktuell einzige Benutzer \texttt{``admin''} hat sozusagen uneingeschränkte Rechte. Durch die Zeile \texttt{" clientRole " : false} ist dieser Benutzer auch nicht auf einen bestimmten Client beschränkt, sondern hat im ganzen Realm auf alles Zugriff.

\subsubsection{Benutzer}
\begin{lstlisting}[
    caption=Keycloak Realm Konfiguration,
    label=lst:realmconfig,
    basicstyle=\ttfamily\small
]
{
"users": [
    {
      "username": "herbsthofer",
      "enabled": true,
      "emailVerified": true,
      "firstName": "Admin",
      "lastName": "Herbsthofer",
      "email": "herbsthofer@example.com",

      "realmRoles": [
        "admin"
      ]
    }
  ]
}
\end{lstlisting}
Der Zurzeit einzige User \texttt{herbsthofer} erhält durch die Zuweisung der admin Rolle, \texttt{'' realmRoles '' : ['' admin '']} alle Administratorrechte.
\subsubsection{Clients}
\begin{lstlisting}[
    caption=Keycloak Realm Client Konfiguration,
    label=lst:realmconfig,
    basicstyle=\ttfamily\small
]
{
   "clients": [
    {
      "clientId": "cesc-frontend",
      "name": "CESC Frontend Application",
      "enabled": true,
      "publicClient": true,
      "protocol": "openid-connect",
      "standardFlowEnabled": true,
      "redirectUris": [
        "https://192.168.20.202/*",
        "http://localhost:4200/*"
      ],
      "webOrigins": [
        "https://192.168.20.202",
        "http://localhost:4200"
      ]
    },
    {
      "clientId": "cesc-backend",
      "name": "CESC Backend API",
      "enabled": true,
      "publicClient": false,
      "bearerOnly": true,
      "protocol": "openid-connect"
    }
  ]
}
\end{lstlisting}
Für unsere Anwendung werden nur zwei Clients benötigt. Der \texttt{cesc-frontend} Client ist der öffentliche Client (\texttt{publicClient: true}) für die Angular Anwendung. Die \texttt{redirectUris} und \texttt{webOrigins} legen die erlaubten URLs für die Produktion (192.168.20.202) und die lokale Entwicklung (localhost:4200) fest. Der \texttt{cesc-backend} Client ist als vertraulicher Client gemacht, da er keine Login Page bereitstellt.(\texttt{publicClient: false}).

\section{Reverse Proxy [Elias Mahr]}
\setauthor{Elias Mahr}

Nginx ist als eigener Docker Container mit nginx.conf implimentiert. Nur der Port 443 ist öffentlich mit Let's Encrypt-Zertifikaten.
Die Ports werden pfadbasiert weitergeleitet:

\begin{itemize}
\item \underline{http://Keycloak:8080 $\to$ location /auth}
\begin{lstlisting}[language=bash, caption=Nginx Konfiguration /auth, label=lst:nginx:/auth]
        location /auth {
            proxy_pass http://keycloak:8080;
            proxy_set_header Host $host;
            proxy_set_header X-Real-IP $remote_addr;
            proxy_set_header X-Forwarded-For $proxy_add_x_forwarded_for;
            proxy_set_header X-Forwarded-Proto https;
            proxy_set_header X-Forwarded-Host $host;
            proxy_set_header X-Forwarded-Port 443;
            proxy_buffer_size 128k;
            proxy_buffers 4 256k;
            proxy_busy_buffers_size 256k;
        }
\end{lstlisting}



\item \underline{http://frontend:80 $\to$ location/}
\begin{lstlisting}[language=bash, caption=Nginx Konfiguration /, label=lst:nginx:/]
        location / {
            proxy_pass http://frontend:80;
        }
\end{lstlisting}
\item \underline{http://backend:5000 $\to$ location /api}
\begin{lstlisting}[language=bash, caption=Nginx Konfiguration /api, label=lst:nginx:/api]
        location /api {
            proxy_pass http://backend:5000;
        }
\end{lstlisting}
\end{itemize}



\section{Angular}
Da das Projekt nicht nur am Desktop, sondern auch auf mobilen Devices zum Einsatz kommen soll, wurde die gesamte Applikation in einem responsive Design entwickelt. 
\setauthor{Elias Mahr}

\subsection{Components}
\subsubsection{start-list-container}

\begin{figure}[H]
    \centering
    \includegraphics[scale=0.2]{pics/start-list-container.png}
    \caption{Startlist Component}
    \label{fig:FrontendButton}
\end{figure}

Die ``StartListContainerComponent'' dient als Startseite der Anwendung.
Beim Öffnen wird eine komplette Übersicht über alle in der Datenbank gespeicherten Baustellen angezeigt.
Die Seite ist in drei wesentliche Abschnitte unterteilt: 
\begin{itemize}
  \item \textbf{Titel}
  
  Ganz ober sieht man sofort den Titel unserer Arbeit(C-E-S-C) mit dem ausgeschriebenen Titel darunter(Container-Energy-Security-Controll).
  \item \textbf{Baustellenübersicht}
  
  Links auf 1/4 der Breite sind alle Baustellen mit ihrem in der Datenbank gespeicherten Namen aufgelistet. Die ausgewählte Baustelle wird immer gelb mit Rahmen markiert, so auch beim Überfahren mit der Maus. In diesem Fall wird das Element auch ein wenig eingerückt. Immer am Ende des Abteils ist das Kalenderelement. Dieses ist blau eingefärbt, damit man es gut von den Baustellen unterscheiden kann.
  \item \textbf{Containerübersicht}
  
  Rechts bis mittig zu 3/4 der Breite ist die Übersicht der jeweiligen Container zu einer Baustelle. Mit dem Wechsel der Baustelle wird auch die Ansicht der Container aktualisiert. Bei dieser Ansicht werden Hauptcontainer optisch hervorgehoben.
  
\begin{figure}[H]
    \centering
    \begin{subfigure}[b]{0.45\textwidth}
        \centering
        \includegraphics[width=0.8\textwidth]{pics/mainContainer.png}
        \caption{Main Container}
        \label{fig:anmeldebutton}
    \end{subfigure}
    \hfill
    \begin{subfigure}[b]{0.45\textwidth}
        \centering
        \includegraphics[width=0.8\textwidth]{pics/normalContainer.png}
        \caption{normaler Container}
        \label{fig:abmeldebutton}
    \end{subfigure}
    \caption{Unterschiedliche Darstellung der Container}
    \label{fig:loginbuttons}
\end{figure}

  Hauptcontainer erkennt man am gelben Rand und der Kennzeichnung ``MAIN'' neben dem Containernamen.
  Mit Klick auf eine Componente wird man zum jeweiligen Dashboard weitergeleitet.
  Der Name(im Screenshot z.B. \texttt{Container\_1}), wird aus der Datenbank
  genommen und kann natürlich auf Wunsch geändert werden.
  Deffaultmäßig ist es zurzeit ``\texttt{Container\_n}''.

\end{itemize}



\subsubsection{Calendar}
\begin{figure}[H]
    \centering
    \includegraphics[scale=0.15]{pics/CalenderEditor.png}
    \caption{Kalenderelement}
    \label{fig:CalenderEditor}
\end{figure}
Die \texttt{Calender} Componente erscheint, wenn man in der \text{start-list-component} keine Baustelle, sondern das blaue \text{Kalenderelement} 
 antippt. Als erstes sieht man sofort den Kalender, bei dem immer das aktuelle Monat ausgewählt ist.
Im Kalender wird zwischen zwei Arten von Urlauben unterschieden.

\begin{figure}[H]
    \centering
    \begin{subfigure}[b]{0.45\textwidth}
        \centering
        \includegraphics[width=0.3\textwidth]{pics/Betriebsurlaub.png}
        \caption{Betriebsurlaub}
        \label{fig:betriebsurlaub}
    \end{subfigure}
    \hfill
    \begin{subfigure}[b]{0.45\textwidth}
        \centering
        \includegraphics[width=0.29\textwidth]{pics/Feiertag.png}
        \caption{Feiertag}
        \label{fig:feiertag}
    \end{subfigure}
    \caption{Unterschied der Urlaube}
    \label{fig:holidays}
\end{figure}

Betriebsurlaub wird gelb markiert und ein Feiertag grau.
An diesen markierten Tagen werden die Container nicht eingeschaltet.
Unter dem Kalender findet man noch eine Liste mit allen Urlauben für das aktuelle Jahr und den Betriebsurlauben.
Die Feiertage werden von einer API geladen. Nach erfolgreicher Anmeldung wird der ``bearbeiten'' Button freigeschaltet und Umänderungen können vorgenommen werden.

\begin{figure}[H]
    \centering
    \begin{subfigure}[b]{0.45\textwidth}
        \centering
        \includegraphics[width=0.4\textwidth]{pics/Freigschalten.png}
        \caption{Freigeschalten}
        \label{fig:betriebsurlaub}
    \end{subfigure}
    \hfill
    \begin{subfigure}[b]{0.45\textwidth}
        \centering
        \includegraphics[width=0.42\textwidth]{pics/lockd.png}
        \caption{ohne Anmeldung}
        \label{fig:feiertag}
    \end{subfigure}
    \caption{Bearbeiten Button}
    \label{fig:bearbeiten}
\end{figure}

\begin{figure}[H]
    \centering
    \includegraphics[scale=0.15]{pics/calender-component.png}
    \caption{Betriebsurlaube bearbeiten}
    \label{fig:Calender}
\end{figure}

Bei Klick auf den Butten kann der Betriebsurlaub bearbeitet werden.
Die neuen Daten werden direkt in die Datenbank gespeichert und sofort vom Cronjob beachtet. Dabei wird natürlich auch geprüft, ob die Daten im korrekten Format eingeben wurden.
Um dies zu erleichtern wird beim Eingeben eines neuen Datums sofort ein kleiner Kalender geöffnet, auf dem ein neues Datum ausgewählt werden kann. Dies ist eine vorgefertigte Funktion von html wenn man die input type date verwendet.

\begin{lstlisting}[
    caption=label im html,
    label=lst:htmlcalender,
    basicstyle=\ttfamily\small
]
{
    <label>Von: <input type="date" [(ngModel)]="modalFrom"></label>
}
\end{lstlisting}


\subsubsection{dashboard}

\begin{figure}[H]
    \centering
    \includegraphics[scale=0.15]{pics/dashboard.png}
    \caption{Dashboard}
    \label{fig:Dashboard}
\end{figure}

Wenn man auf den Button \texttt{Dashboard} in der \texttt{start-list-componente} bei einem Container klickt, kommt man zur \texttt{dashboard} Komponente. 
Diese Hauptfunktion der Applikation ist für jeden Container individuell abhängig, von den jeweiligen Daten.
Der Aufbau ist jedoch überall gleich. Die Componente ist so gestaltet, dass sie jederzeit erweitert werden kann. 
Für jeden Sensor ist ein eigener abgetrennter Abschnitt auf der Webseite. Bei Erweiterung des Systems durch die Firma
kann ein weiterer Abschnitt jederzeit leicht hinzugefügt werden, ohne Rücksicht auf die anderen Componenten nehmen zu müssen.

Ganz oben in der Componente sieht man die Überschrift mit dem Namen des Containers. Der erste Abschnitt ist eine übersichtliche Temperaturanzeige von Innen- und Außensensor.
Fettgedruckt ist die Temperatur, da das die wichtigste Anzeige im Container ist. Gleich darunter steht etwas kleiner die Luftfeuchtigkeit und abschließend noch die Anzeige der übrigen Batterie, um rechtzeitig zu erkennen, wann die Batterien gewechselt werden sollten.
Ein detaillierter Verlauf kann dann noch mit dem Button ``Verlauf'' angezeigt werden. Dazu aber dann in dieser Componente mehr.

\subsubsection{door-logs}

\begin{figure}[H]
    \centering
    \begin{subfigure}[b]{0.45\textwidth}
        \centering
        \includegraphics[width=1\textwidth]{pics/doorLogs.png}
        \label{fig:doorLogsUp}
    \end{subfigure}
    \hfill
    \begin{subfigure}[b]{0.45\textwidth}
        \centering
        \includegraphics[width=1\textwidth]{pics/doorLogsDown.png}
        \label{fig:doorLogsdown}
    \end{subfigure}
    \caption{Door Logs Componente}
    \label{fig:doorLogs}
\end{figure}

\subsubsection{plotly-chart}




\begin{itemize}
\item can- und Logout-Buttonslender-editor
\item dashboard
\item door-logs
\item kalender
\item plotly-chart
\item start-list-container
\end{itemize}

\subsubsection{Services}








Siehe tolle Daten in Tab. \ref{tab:impl:data}.
\setauthor{Elias Mahr}


Siehe und staune in Abb. \ref{fig:impl:knuth}.

Dann betrachte den Code in Listing \ref{lst:impl:foo}.

\begin{table}[b]
    \centering
    \begin{tabular}{|lcc|}
    \hline
              & \textbf{Regular Customers} & \textbf{Random Customers} \\ \hline
    Age       & 20-40                      & \textgreater{}60          \\ \hline
    Education & university                 & high school               \\ \hline
    \end{tabular}
    \caption{Ein paar tabellarische Daten}
    \label{tab:impl:data}
\end{table}


