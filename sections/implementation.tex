\section{Keycloak [Elias Mahr]}
\setauthor{Elias Mahr}

\subsection{Einbindung ins Frontend}

\begin{figure}[H]
    \centering
    \includegraphics[scale=0.5]{pics/FrontendButton.png}
    \caption{Button im Frontend}
    \label{fig:FrontendButton}
\end{figure}

Um die Webapp möglichst unkompliziert und Benutzerfreundlich zu halten ist der Button wie im Screenshot rechts oben in der Ecke platziert. Unabhängig vom Scroll-Zustand oder der aktuell ausgewählten Page bleibt er auch immer dort, um sich sofort und schnell anmelden zu können.

\begin{figure}[H]
    \centering
    \begin{subfigure}[b]{0.45\textwidth}
        \centering
        \includegraphics[width=0.5\textwidth]{pics/anmelden.png}
        \caption{Anmelde-Button}
        \label{fig:anmeldebutton}
    \end{subfigure}
    \hfill
    \begin{subfigure}[b]{0.45\textwidth}
        \centering
        \includegraphics[width=0.5\textwidth]{pics/abmelden.png}
        \caption{Abmelde-Button}
        \label{fig:abmeldebutton}
    \end{subfigure}
    \caption{Login- und Logout-Buttons}
    \label{fig:loginbuttons}
\end{figure}

Das Icon stammt aus der Font Awesome Bibliothek:
\\
\lstinline[language=HTML]{<i class="fas fa-sign-in-alt"></i>}
\\
Bei Klick darauf wird man weitergeleitet zum Pfad:\\ \texttt{https://192.168.20.202/auth/realms/cesc/protocol/openid-connect\\/auth?client\_id=cesc-frontend\&redirect\_uri=https...} 
Hier kommt dann die Standart Keycloak Login Page. Nach erfolgreicher Anmeldung bekommt der User seinen Token, wird automatisch wieder umgeleitet zur vorher besuchten Seite und es werden alle Funktionen der Webseite freigeschalten.

\subsection{das Realm}
Ein Realm in Keycloak ist ein isolierter Sicherheitsbereich. Er umfasst Benutzer, Rollen, Clients und Policies für eine bestimmte Anwendung. Wir haben ein Realm namens ``cesc'' erstellt. Es definiert die gesamte Identifikation und Authorisierung für die Weboberfläche. Zurzeit ist es möglichst klein gehalten, jedoch vollkommen ausreichend. Ein Admin User(herbsthofer) steuert alles, in Zukunft kann man jedoch leicht noch andere Benutzer/Rollen hinzu fügen.
\subsubsection{Realm Einstellungen}

\begin{lstlisting}[
    caption=Keycloak Realm Konfiguration,
    label=lst:realmconfig,
    basicstyle=\ttfamily\small
]
{
  "realm": "cesc",
  "enabled": true,
  "sslRequired": "external",
  "registrationAllowed": false,
  "loginWithEmailAllowed": true,
  "duplicateEmailsAllowed": false,
  "resetPasswordAllowed": true,
  "editUsernameAllowed": false,
  "bruteForceProtected": true
}
\end{lstlisting}
In den wichtigsten Einstellungen wird unter anderem definiert das für alle externen Verbindungen SSL vorgeschrieben ist.
Die Zeile \texttt{bruteForceProtected: true} ist eine einfache Absicherung gegen Brute Force Angriffe von Keycloak. Durch sie wird ein Account nach mehreren fehlgeschlagenen Anmeldungen Temporär gesperrt.
Um zukünftige Admins der Webseite eine leichte und unkomplizierte Anmeldung zu ermöglichen ist das einloggen mit der email Adresse auch erlaubt.

\subsubsection{Rollen}
\begin{lstlisting}[
    caption=Keycloak Realm Konfiguration,
    label=lst:realmconfig,
    basicstyle=\ttfamily\small
]
{
  "roles": {
    "realm": [
      {
        "name": "admin",
        "description": "Administrator role for CESC application",
        "composite": false,
        "clientRole": false
      }
    ]
  }
}
\end{lstlisting}
Die wie oben erwähnt zurzeit einzige Rolle \texttt{admin} hat kurtzgesatz Rechte auf alles. Durch die Zeile \texttt{" clientRole " : false} is sie zum Beispiel auch nicht auf einen Gewissen Client beschränk sondern hat im ganzen Realm auf alles Zugriff.

\subsubsection{Benutzer}
\begin{lstlisting}[
    caption=Keycloak Realm Konfiguration,
    label=lst:realmconfig,
    basicstyle=\ttfamily\small
]
{
"users": [
    {
      "username": "herbsthofer",
      "enabled": true,
      "emailVerified": true,
      "firstName": "Admin",
      "lastName": "Herbsthofer",
      "email": "herbsthofer@example.com",

      "realmRoles": [
        "admin"
      ]
    }
  ]
}
\end{lstlisting}
Der Zurzeit einzige User \texttt{herbsthofer} erhält durch die Zuweisung der admin Rolle, \texttt{'' realmRoles '' : ['' admin '']} alle Administrator rechte.
\subsubsection{Clients}
\begin{lstlisting}[
    caption=Keycloak Realm Client Konfiguration,
    label=lst:realmconfig,
    basicstyle=\ttfamily\small
]
{
   "clients": [
    {
      "clientId": "cesc-frontend",
      "name": "CESC Frontend Application",
      "enabled": true,
      "publicClient": true,
      "protocol": "openid-connect",
      "standardFlowEnabled": true,
      "redirectUris": [
        "https://192.168.20.202/*",
        "http://localhost:4200/*"
      ],
      "webOrigins": [
        "https://192.168.20.202",
        "http://localhost:4200"
      ]
    },
    {
      "clientId": "cesc-backend",
      "name": "CESC Backend API",
      "enabled": true,
      "publicClient": false,
      "bearerOnly": true,
      "protocol": "openid-connect"
    }
  ]
}
\end{lstlisting}
Für unsere Anwendung werden nur zwei Clients benötigt. Der \texttt{cesc-frontend} Client ist der öffentliche Client (\texttt{publicClient: true}) für die Angular Anwendung. Die \texttt{redirectUris} und \texttt{webOrigins} legen die erlaubten URLs für die Produktion (192.168.20.202) und die lokale Entwicklung (localhost:4200) fest. Der \texttt{cesc-backend} Client ist als vertraulicher Client gemacht, da er keine Login Page bereitstellt.(\texttt{publicClient: false}).

\section{Reverse Proxy [Elias Mahr]}
\setauthor{Elias Mahr}

Nginx ist als eigener Docker Container mit nginx.conf implimentiert. Nur der Port 443 ist öffentlich mit Let's Encrypt-Zertifikaten.
Die Ports werden Pfadbasiert weitergeleitet:

\begin{itemize}
\item \underline{http://Keycloak:8080 $\to$ location /auth}
\begin{lstlisting}[language=bash, caption=Nginx Konfiguration /auth, label=lst:nginx:/auth]
        location /auth {
            proxy_pass http://keycloak:8080;
            proxy_set_header Host $host;
            proxy_set_header X-Real-IP $remote_addr;
            proxy_set_header X-Forwarded-For $proxy_add_x_forwarded_for;
            proxy_set_header X-Forwarded-Proto https;
            proxy_set_header X-Forwarded-Host $host;
            proxy_set_header X-Forwarded-Port 443;
            proxy_buffer_size 128k;
            proxy_buffers 4 256k;
            proxy_busy_buffers_size 256k;
        }
\end{lstlisting}
\item \underline{http://frontend:80 $\to$ location/}
\begin{lstlisting}[language=bash, caption=Nginx Konfiguration /, label=lst:nginx:/]
        location / {
            proxy_pass http://frontend:80;
        }
\end{lstlisting}
\item \underline{http://backend:5000 $\to$ location /api}
\begin{lstlisting}[language=bash, caption=Nginx Konfiguration /api, label=lst:nginx:/api]
        location /api {
            proxy_pass http://backend:5000;
        }
\end{lstlisting}
\end{itemize}



\section{Angular}
\setauthor{Elias Mahr}

\subsection{Components}
\subsubsection{start-list-container}

\begin{figure}[H]
    \centering
    \includegraphics[scale=0.2]{pics/start-list-container.png}
    \caption{Startlist Component}
    \label{fig:FrontendButton}
\end{figure}

Die ``StartListContainerComponent'' dient als Startseite der Anwendung.
Dort wird bei öffnen schon eine Komplette Übersicht über alles Verfügbaren Baustellen in der Datenbank gegeben.
Die Seite ist in 3 Wesentliche Abschnitte unterteilt: 
\begin{itemize}
  \item \textbf{Titel}
  
  Ganz ober sieht man sofort den Titel unserer Arbeit(C-E-S-C) mit dem  Vollausgeschriebenen Titel darunter(Container-Energy-Security-Controll).
  \item \textbf{Baustellen Übersicht}
  
  Links auf 1/4 der breite sind alle Baustellen aufgelistet, ganz unten die Kalender Komponente. Bei der Auflistung werden die Namen der Baustellen aus der Datenbank genommen. Die ausgewählte Baustelle wird immer Gelb mit Ramen markiert, so auch wenn die Maus darüber Hovert. In diesem Fall wird das Element auch ein wenig eingerückt.
  \item \textbf{Container Übersicht}
  
  Rechts bis Mittig zu 3/4 der Breite ist die Übersicht der Jeweiligen Container zu einer Baustelle. Die Angezeigten Container werden bei Klicken auf eine anderen Baustelle gewechselt. Optisch wird unterschieden zwischen Main-Container und normalen Container. 
  
\begin{figure}[H]
    \centering
    \begin{subfigure}[b]{0.45\textwidth}
        \centering
        \includegraphics[width=0.8\textwidth]{pics/mainContainer.png}
        \caption{Main Container}
        \label{fig:anmeldebutton}
    \end{subfigure}
    \hfill
    \begin{subfigure}[b]{0.45\textwidth}
        \centering
        \includegraphics[width=0.8\textwidth]{pics/normalContainer.png}
        \caption{normaler Container}
        \label{fig:abmeldebutton}
    \end{subfigure}
    \caption{unterschied der Container}
    \label{fig:loginbuttons}
\end{figure}

  Durch Gelben Rand und der Kennzeichnung ``MAIN'' neben dem Container Namen.
  Bei Klick auf eine Componente wird man zum jeweiligen Dashboard weiter geleitet.
  Der Name(im Screenshot z.B. \texttt{Container\_1}), wird aus der Datenbank
  genommen und kann natürlich auf Wunsch geändert werden.
  Deffaultmäßig ist es zurzeit ``\texttt{Container\_n}''.

\end{itemize}

\subsubsection{calender-editor}

\subsubsection{dashboard}

\subsubsection{door-logs}

\subsubsection{kalender}

\subsubsection{login-modal}

\subsubsection{navbar}

\subsubsection{plotly-chart}




\begin{itemize}
\item can- und Logout-Buttonslender-editor
\item dashboard
\item door-logs
\item kalender
\item login-modal
\item navbar
\item plotly-chart
\item start-list-container
\end{itemize}

\subsubsection{Services}








Siehe tolle Daten in Tab. \ref{tab:impl:data}.
\setauthor{Elias Mahr}


Siehe und staune in Abb. \ref{fig:impl:knuth}.

Dann betrachte den Code in Listing \ref{lst:impl:foo}.

\begin{table}[b]
    \centering
    \begin{tabular}{|lcc|}
    \hline
              & \textbf{Regular Customers} & \textbf{Random Customers} \\ \hline
    Age       & 20-40                      & \textgreater{}60          \\ \hline
    Education & university                 & high school               \\ \hline
    \end{tabular}
    \caption{Ein paar tabellarische Daten}
    \label{tab:impl:data}
\end{table}


