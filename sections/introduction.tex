\section{Ausgangslage [Elias Mahr]}
\setauthor{Elias Mahr}

\begin{wrapfigure}{r}{0.4\textwidth}
    \centering
    \includegraphics[width=0.38\textwidth]{pics/HerbsthoferImage.png}
    \caption{Herbsthofer Logo}
    \label{fig:container}
\end{wrapfigure}
Das seit 1870 Familien geführte Unternehmen Herbsthofer plant, liefert, montiert und wartet HKLS-Anlagen für Industrie, Gesundheitswesen und Forschung. Beispielhafte Projekte wären die dm Zentrale, das MSD Krems und die Techbase in Linz. Das Unternehmen nutzt 3D-Planung und viele moderne, digitale Tools zur Kosten- und Qualitätskontrolle. Nationale und internationale Projekte werden vom Hauptsitz Linz aus umgesetzt. Hier mehr dazu: \url{https://www.herbsthofer.at/leistungsspektrum}

\section{Ist-Zustand [Elias Mahr]}
\setauthor{Elias Mahr}
Das Unternehmen betreibt zahlreiche Baustellen im In- und Ausland. Für die Dauer der Projekte werden auf diesen Baustellen Containeranlagen genutzt, die als temporäres Büro, Aufenthalts- und Lagerräume dienen. Die Container sind je nach Nutzung mit W-Lan, Heizkörpern, Kühlsystem und Beleuchtung ausgestattet. Um einen durchgehenden Betrieb und geeignete Arbeitsbedingungen für das Personal zu gewährleisten wird durchgehend geheizt/gekühlt.

\section{Problem [Leopold Mistelberger]}
\setauthor{Leopold Mistelberger}
Ein Baustellencontainer ohne Überwachung produziert unnötige Energiekosten. Zudem fehlt eine Kontrolle über den Zustand des Containers. Häufig wird in einem Container eigebrochen, welche unbemerkt bleiben, und der Täter nicht identifiziert werden kann.

\section{Aufgabenstellung [Leopold Mistelberger]}
\setauthor{Leopold Mistelberger}
Ziel dieser Diplomarbeit ist die Entwicklung des Systems CESC zur Überwachung und Steuerung eines Baustellencontainers. Das System soll Temperatur, Luftfeuchtigkeit, Türstatus und Anwesenheit erfassen, visualisieren und bei kritischen Ereignissen Benachrichtigungen senden. CESC soll dazu beitragen, Energiekosten zu senken und Einbrüche frühzeitig zu erkennen.


