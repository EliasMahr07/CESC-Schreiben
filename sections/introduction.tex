\section{Ausgangslage [Elias Mahr]}
\setauthor{Elias Mahr}

\begin{wrapfigure}{r}{0.4\textwidth}
    \centering
    \includegraphics[width=0.38\textwidth]{pics/HerbsthoferImage.png}
    \caption{Herbsthofer Logo}
    \label{fig:container}
\end{wrapfigure}
Das seit 1870 familiengeführte Unternehmen Herbsthofer plant, liefert, montiert und wartet Heizungs-, Kühlungs-, Lüftungs- und Sanitäranlagen (kurz HKLS-Anlagen) für Industrie, Gesundheitswesen und Forschung. Beispielhafte Projekte wären die dm Zentrale, das MSD Krems und die Techbase in Linz. Das Unternehmen nutzt 3D-Planung und moderne, digitale Tools zur Kosten- und Qualitätskontrolle. Nationale und internationale Projekte werden vom Hauptsitz Linz aus umgesetzt. Hier mehr dazu: \url{https://www.herbsthofer.at/leistungsspektrum}

\section{Ist-Zustand [Elias Mahr]}
\setauthor{Elias Mahr}
Das Unternehmen betreibt zahlreiche Baustellen im In- und Ausland. Für die Dauer der Projekte werden auf diesen Baustellen Containeranlagen genutzt, die als temporäres Büro, Aufenthalts- und Lagerräume dienen. Die Container sind je nach Nutzung mit W-Lan, Heizkörpern, Kühlsystem und Beleuchtung ausgestattet. Um einen durchgehenden Betrieb und geeignete Arbeitsbedingungen für das Personal zu gewährleisten wird durchgehend geheizt/gekühlt, wobei die Steuerung manuell vor Ort erfolgt.

\section{Problem [Leopold Mistelberger]}
\setauthor{Leopold Mistelberger}
Baustellencontainer werden häufig ohne technische Überwachung betrieben, wodurch hohe Energiekosten entstehen, da hauptsächlich die Heizung unabhängig von der tatsächlichen Nutzung betrieben wird. Eine Kontrolle über den Zustand des Containers ist nicht gegeben, weshalb die Firma keinen Zugriff auf aktuelle Informationen über Heizung, Raumklima, geöffnete Türen, Nutzung, Anwesenheit, Temperatur, Luftfeuchtigkeit und Lichtstatus hat.

Dieser Mangel erschwert dem Betrieb eine effiziente Steuerung der Baustellencontainer und führt zu erhöhten Betriebskosten. Darüber hinaus stellen Container ein gewisses Sicherheitsrisiko dar. Einbrüche bleiben oftmals unbemerkt, da weder eine automatische Erkennung noch eine sofortige Benachrichtigung erfolgt. In solchen Fällen ist eine zeitnahe Reaktion nicht möglich, und Vorfälle können nur im Nachhinein nachvollzogen werden, was die Suche nach dem Täter erheblich erschwert.

\section{Aufgabenstellung [Leopold Mistelberger]}
\setauthor{Leopold Mistelberger}
Ziel dieser Diplomarbeit ist die Entwicklung des Systems CESC zur Überwachung und Steuerung eines Baustellencontainers. Das System soll Temperatur, Luftfeuchtigkeit, Türstatus und Anwesenheit erfassen und visualisieren sowie bei kritischen Ereignissen, wie etwa einem Einbruch, Benachrichtigungen senden.

Zusätzlich soll CESC die Steuerung der Heizung, der Klimatisierung und der Beleuchtung ermöglichen. Bei kritischen oder sicherheitsrelevanten Ereignissen sollen automatisch Benachrichtigungen versendet werden. Durch den Einsatz von CESC sollen Energiekosten gesenkt und die Sicherheit auf Baustellen erhöht werden.


