\section{Ausgangslage [Elias Mahr]}
\setauthor{Elias Mahr}

\begin{wrapfigure}{r}{0.4\textwidth}
    \centering
    \includegraphics[width=0.38\textwidth]{pics/HerbsthoferImage.png}
    \caption{Herbsthofer Logo}
    \label{fig:container}
\end{wrapfigure}
Das seit 1870 familiengeführte Unternehmen Herbsthofer plant, liefert, montiert und wartet Heizungs-, Kühlungs-, Lüftungs- und Sanitäranlagen (kurz HKLS-Anlagen) für Industrie, Gesundheitswesen und Forschung. Beispielhafte Projekte wären die dm Zentrale, das MSD Krems und die Techbase in Linz. Das Unternehmen nutzt 3D-Planung und moderne, digitale Tools zur Kosten- und Qualitätskontrolle. Nationale und internationale Projekte werden vom Hauptsitz Linz aus umgesetzt. Hier mehr dazu: \url{https://www.herbsthofer.at/leistungsspektrum}

\section{Ist-Zustand [Elias Mahr]}
\setauthor{Elias Mahr}
Das Unternehmen betreibt zahlreiche Baustellen im In- und Ausland. Für die Dauer der Projekte werden auf diesen Baustellen Containeranlagen genutzt, die als temporäres Büro, Aufenthalts- und Lagerräume dienen. Die Container sind je nach Nutzung mit W-Lan, Heizkörpern, Kühlsystem und Beleuchtung ausgestattet. Um einen durchgehenden Betrieb und geeignete Arbeitsbedingungen für das Personal zu gewährleisten wird durchgehend geheizt/gekühlt, wobei die Steuerung manuell vor Ort erfolgt.

\section{Problem [Leopold Mistelberger]}
\setauthor{Leopold Mistelberger}
Baustellencontainer werden häufig ohne technische Überwachung betrieben, wodurch hohe Energiekosten entstehen, da hauptsächlich die Heizung unabhängig von der tatsächlichen Nutzung betrieben wird. Eine Kontrolle über den Zustand des Containers ist nicht gegeben, weshalb die Firma keinen Zugriff auf aktuelle Informationen über Heizung, Raumklima, geöffnete Türen, Nutzung, Anwesenheit, Temperatur, Luftfeuchtigkeit und Lichtstatus hat.

Dieser Mangel erschwert dem Betrieb eine effiziente Steuerung der Baustellencontainer und führt zu erhöhten Betriebskosten. Darüber hinaus stellen Container ein gewisses Sicherheitsrisiko dar. Einbrüche bleiben oftmals unbemerkt, da weder eine automatische Erkennung noch eine sofortige Benachrichtigung erfolgt. In solchen Fällen ist eine zeitnahe Reaktion nicht möglich, und Vorfälle können nur im Nachhinein nachvollzogen werden, was die Suche nach dem Täter erheblich erschwert.

\section{Aufgabenstellung [Leopold Mistelberger]}
\setauthor{Leopold Mistelberger}
Ziel dieser Diplomarbeit ist die Entwicklung des Systems CESC zur Überwachung, Steuerung und erhöhten Sicherheit aller Baustellencontainer der Firma Herbsthofer. Das System soll auch dazu beitragen, sowohl die Betriebssicherheit als auch die Energieeffizienz auf Baustellen nachhaltig und automatisiert zu verbessern.

Im Rahmen der Arbeit soll ein verteiltes IoT-basiertes System entwickelt werden, das sowohl in Aufenthalts- als auch Lagercontainern implementiert wird. Innerhalb eines Containers werden verschiedene Umfeld- und Zustandsdaten erfasst, verarbeitet und visualisiert. Dazu zählen Innen-/Außentemperatur, Innen-/Außenfeuchtigkeit, Türzustand sowie die Erkennung einer Anwesenheit.

<<<<<<< HEAD
\section{Zustandsdiagramm}

\begin{figure}[H]
    \centering
    \includegraphics[scale=0.15]{pics/Events.png}
    \caption{Zustandsdiagram}
    \label{fig:Zustandsdiagram}
\end{figure}
=======
Ein weiterer wichtiger Teil des Systems ist die aktive Steuerung von Heizung, Klimatisierung und Licht. Ziel ist es, den Energieverbrauch durch automatisierte und zeitgesteuerte Steuerungsmechanismen zu minimieren, ohne dabei die Arbeiter zu behindern oder die Funktionsfähigkeit des Containers einzuschränken. Bei Sonderfällen soll auch ein manueller Eingriff über eine Benutzeroberfläche möglich sein.
>>>>>>> eb54245655fab54e7a4d4652ee6af670f364510a

Ein besonderer Fokus dieser Arbeit liegt auf der Sicherheit. In einem Container sollen zusätzlich Kameras verbaut werden, die eine Aufnahme starten, wenn ein unbefugtes Eintreten außerhalb der Arbeitszeit stattfindet. In solchen Fällen soll automatisch der Geschäftsführer per E-Mail oder SMS eine Benachrichtigung erhalten, um ein rasches Eingreifen zu ermöglichen.
