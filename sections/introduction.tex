\section{Ausgangslage}
\setauthor{Elias Mahr}
Die Firma Herbsthofer plant, liefert, montiert und wartet HKLS-Anlagen und Reinräume für Industrie, Gesundheitswesen und Forschung. Das Unternehmen nutzt 3D-Planung und digitale Baustellenlogistik zur Kosten- und Qualitätskontrolle. Nationale und internationale Projekte werden vom Hauptsitz Linz aus umgesetzt. 

\section{Ist-Zustand}
\setauthor{Elias Mahr}
Das Unternehmen betreibt zahlreiche Baustellen im In- und Ausland. Für die Dauer der Projekte werden auf diesen Baustellen Containeranlagen genutzt, die als Büro-, Aufenthalts- und Lagerräume dienen. Diese Container werden rund um die Uhr beheizt, um einen durchgehenden Betrieb und geeignete Arbeitsbedingungen für das Personal zu gewährleisten.

\section{Problem}
\setauthor{Leopold Mistelberger}
Ein Baustellencontainer ohne Überwachung produziert unnötige Energiekosten. Zudem fehlt eine Kontrolle über den Zustand des Containers. Häufig wird in einem Container eigebrochen, welche unbemerkt bleiben, und der Täter nicht identifiziert werden kann.

\section{Aufgabenstellung}
\setauthor{Leopold Mistelberger}
Ziel dieser Diplomarbeit ist die Entwicklung des Systems CESC zur Überwachung und Steuerung eines Baustellencontainers. Das System soll Temperatur, Luftfeuchtigkeit, Türstatus und Anwesenheit erfassen, visualisieren und bei kritischen Ereignissen Benachrichtigungen senden. CESC soll dazu beitragen, Energiekosten zu senken und Einbrüche frühzeitig zu erkennen.


